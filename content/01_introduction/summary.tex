\section{Zusammenfassung}
Im diesem Kapitel wurde die Motivation der geplanten Anwendung dargelegt und zwei Anwendungsbeispiele betrachtet. Anschließend wurde dargestellt, welche Ziele erreicht werden sollen und angeschnitten wie diese realisiert werden sollen. Dies wird in der weiteren Arbeit genauer beschrieben. 

In Kapitel \ref{chap:basics} wird auf die Grundlagen wie Themenmodelle und Zentralitätsmaße eingegangen, die in der Literatur schon erforscht wurden und hier nicht weiter entwickelt bzw. nur unverändert benutzt werden. Kapitel \ref{chap:relatedWork} stellt Arbeiten vor, die mit ähnlichen Ansätzen arbeiten, wie sie in der Diplomarbeit verfolgt werden. Das \ref{chap:workflowData}. Kapitel stellt die verwendeten Daten und die Vorarbeit, die für diese Daten nötig ist, dar. Insbesondere wird darauf eingegangen, welche Daten zum Training des Themenmodells benutzt und welche Daten als Textsequenz benutzt werden. 

Der eigentliche entwickelte Algorithmus, bzw. die Applikation zur Erkennung und Bewertung der Themen wird in Kapitel \ref{chap:workflowTime} erläutert. Dort werden die einzelnen Schritte dargestellt, wie man von einem sequentiellen Text zu einer Darstellung von thematischen Verläufen kommt. Zusätzlich werden Methoden zur Evaluation des Algorithmus entwickelt. Die Ergebnisse der durchgeführten Experimente und die Bewertung derselben wird in Kapitel \ref{chap:results} veranschaulicht. Im letzten Kapitel wird noch ein kurzer Ausblick auf verschiedene Anwendungen der entwickelten Methode gegeben und die Methode und ihre Ergebnisse kritisch hinterfragt. 