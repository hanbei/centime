\section{Zusammenfassung} 
In den letzten beiden Abschnitten wurden die in der Diplomarbeit verwendeten, grundlegenden Techniken vorgestellt. Es wurde gezeigt, wie aus Texten Themen extrahiert werden und es wurden die verwendeten Zentralitätsindizes vorgestellt. Im Abschnitt über Themenmodelle wurde die verwendete mathematische Notation für Korpora, Dokumente, Wörter und Themenverteilungen eingeführt. Es wurde mathematisch hergeleitet, wie der Algorithmus zur Erkennung von Themen funktioniert und auch wie er implementiert werden kann. Spezielles Augenmerk wurde auf die statistische Inferenz von Themen aus Texten und die Inferenz von Themen für neue Texte gelegt, da dies die Hauptanwendung der Diplomarbeit ist. Es wurde aber auch kurz auf das generative Modell eingegangen, welches dazu benutzt werden kann, aus einem gelernten Themenmodell wiederum Dokumente zu generieren. Dies wird später bei der Evaluation der Themenverläufe eine Rolle spielen.

Im Abschnitt über Zentralitätsindizes wurden die verwendeten Zentralitätsindizes vorgestellt. Es wurde die notwendigen Definitionen für Graphen erläutert und versucht eine mathematische Definition für Zentralitätsindizes wiederzugeben. Die einzelnen Zentralitätsindizes wurden durch Beispiele für ihre Anwendung beschrieben und anschließend wurde die mathematische Formulierung dargestellt. Insbesondere wurden die Schwachstellen der Zentralitätsindizes erläutert, die später berücksichtigt werden müssen. 